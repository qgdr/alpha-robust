% SIAM Article Template
\documentclass[review,hidelinks,onefignum,onetabnum]{siamart220329}

% Information that is shared between the article and the supplement
% (title and author information, macros, packages, etc.) goes into
% ex_shared.tex. If there is no supplement, this file can be included
% directly.

% SIAM Shared Information Template
% This is information that is shared between the main document and any
% supplement. If no supplement is required, then this information can
% be included directly in the main document.


% Packages and macros go here
\usepackage{lipsum}
\usepackage{amsfonts}
\usepackage{graphicx}
\usepackage{epstopdf}
\usepackage{algorithmic}
\usepackage{amsmath, amssymb}
\ifpdf
  \DeclareGraphicsExtensions{.eps,.pdf,.png,.jpg}
\else
  \DeclareGraphicsExtensions{.eps}
\fi

% Add a serial/Oxford comma by default.
\newcommand{\creflastconjunction}{, and~}

% Used for creating new theorem and remark environments
\newsiamremark{remark}{Remark}
\newsiamremark{hypothesis}{Hypothesis}
\crefname{hypothesis}{Hypothesis}{Hypotheses}
\newsiamthm{claim}{Claim}

% Sets running headers as well as PDF title and authors
\headers{An Example Article}{D. Doe, P. T. Frank, and J. E. Smith}

% Title. If the supplement option is on, then "Supplementary Material"
% is automatically inserted before the title.
\title{A superlinear Method for Space-Time fractional diffusion equation with the low regularity solution\thanks{Submitted to the editors DATE.
\funding{This work was funded by the Fog Research Institute under contract no.~FRI-454.}}}

% Authors: full names plus addresses.
\author{Jianxing Han}%\thanks{Imagination Corp., Chicago, IL 
%   (\email{ddoe@imag.com}, \url{http://www.imag.com/\string~ddoe/}).}
% \and Paul T. Frank\thanks{Department of Applied Mathematics, Fictional University, Boise, ID 
%   (\email{ptfrank@fictional.edu}, \email{jesmith@fictional.edu}).}
% \and Jane E. Smith\footnotemark[3]}

\usepackage{amsopn}
\DeclareMathOperator{\diag}{diag}


%%% Local Variables: 
%%% mode:latex
%%% TeX-master: "ex_article"
%%% End: 


% Optional PDF information
\ifpdf
\hypersetup{
  pdftitle={An Example Article},
  pdfauthor={D. Doe, P. T. Frank, and J. E. Smith}
}
\fi

% The next statement enables references to information in the
% supplement. See the xr-hyperref package for details.

\externaldocument[][nocite]{ex_supplement}

% FundRef data to be entered by SIAM
%<funding-group specific-use="FundRef">
%<award-group>
%<funding-source>
%<named-content content-type="funder-name"> 
%</named-content> 
%<named-content content-type="funder-identifier"> 
%</named-content>
%</funding-source>
%<award-id> </award-id>
%</award-group>
%</funding-group>

\begin{document}

\maketitle

% REQUIRED
\begin{abstract}
This is an example SIAM \LaTeX\ article. This can be used as a
template for new articles.  Abstracts must be able to stand alone
and so cannot contain citations to the paper's references,
equations, etc.  An abstract must consist of a single paragraph and
be concise. Because of online formatting, abstracts must appear as
plain as possible. Any equations should be inline.
\end{abstract}

% REQUIRED
\begin{keywords}
example, \LaTeX
\end{keywords}

% REQUIRED
\begin{MSCcodes}
68Q25, 68R10, 68U05
\end{MSCcodes}

\section{Introduction}
We study
\begin{equation}
  \frac{\partial u}{\partial t} +  (-\Delta)^{\frac{\alpha}{2}} u = f(x,t), \quad x \in \Omega, t\in (0,T].
\end{equation}

Implicity scheme: Let $\tau = \frac{T}{M}$, $U^{n}, F^{n} \in \mathbb{R}^{2N-1}$,
\begin{equation}
  \frac{U^{n+1} - U^{n} }{\tau} + AU^{n+1} = F^{n+1}.
\end{equation}
Then
\begin{equation}
  (I+\tau A)E^{n+1} = E^{n} + \tau R^{n+1}.
\end{equation}
We will prove the convergence of this scheme.

\section{Property of $A$}
\begin{lemma}
  The stiffness matrix $A$ has the following properties:
  \begin{enumerate}
    \item The eigenvalues of $A$ are positive real numbers.
    \item $A$ is positive definite, which means that the eigenvalues of $\frac{A+A^T}{2}$ are positive.
    \item The eigenvectors of $A$ are orthogonal in space where $\langle u,v \rangle := uHv$, 
    where $H := \text{diag}\left( \frac{h_i+h_{i+1}}{2} \right)$.
  \end{enumerate}
\end{lemma}
\begin{proof}
  Since
  \begin{equation}
    A = H^{-1} D = H^{-1/2} H^{-1/2} D H^{-1/2} H^{1/2} ,
  \end{equation}
  where $H^{-1/2} D H^{-1/2}$ is symmetric positive definite, $H^{-1/2} D H^{-1/2} = U\Lambda U^T$.
  Thus,
  \begin{equation}
    A = H^{-1/2} U \Lambda U^T H^{1/2} = (H^{-1/2} U) \Lambda (H^{-1/2} U)^{-1}.
  \end{equation}

  The eigenvectors of $A$ form an orthogonal basis of the Hilbert space defined by $\langle u,v \rangle := uHv$. 
  Let $v_i = H^{-1/2} u_i$ be an eigenvector of $A$ with eigenvalue $\lambda_i$.
\end{proof}
\textcolor{red}{We need to prove $\lambda_1 > c$ for some positive constant $c$.}




\section{Convergence}
\textcolor{red}{
We prove the shceme in convergence in the meaning of $L^2$ norm.
$\|v\|_{2,h} := v_i^T H v_i$ is bounded.
\begin{equation}
  \begin{aligned}
    E^{n} &= (I+\tau A)^{-1}E^{n-1} + (I+\tau A)^{-1}\tau R^{n} \\
    &= (I+\tau A)^{-n}E^{0} + \sum_{k=1}^{n} (I+\tau A)^{-k} \tau R^{n-k+1}
  \end{aligned}
\end{equation}
Assume that $|R^{n}| \le C h^{\min\{r\alpha/2, 2\}}(x_i^{-\alpha} + (2T-x_i)^{-\alpha}) + C(r-1) h^2 (T - \delta(x_i) + h_N)^{1-\alpha} + C \tau^{?}$
\begin{equation}
  \begin{aligned}
    (I+\tau A)^{-k} \tau R^{n-k+1} &= (\tau A) (I+\tau A)^{-k} (\tau A)^{-1} \tau R^{n-k+1} \\
    &= (\tau A) (I+\tau A)^{-k} (A^{-1} R^{n-k+1})
  \end{aligned}
\end{equation}
Since $A^{-1} R^{n-k+1}$ is bounded, $\|A^{-1} R^{n-k+1}\|_{2}$ is bounded. 
Depose it by the basis $v_i$, we have
\begin{equation}
  \begin{aligned}
    \|(I+\tau A)^{-k} \tau R^{n-k+1}\|_2 
    &\le \frac{\tau \lambda_i}{(1+\tau \lambda_i)^{k}} \|A^{-1} R^{n-k+1}\|_{2}
  \end{aligned}
\end{equation}
Then
\begin{equation}
  \begin{aligned}
    \|E^{n}\|_{2} &\le \|E_0\|_{2} + \sum_{k=1}^{n} \frac{\tau \lambda_1}{(1+\tau \lambda_1)^{k}} \|A^{-1} R^{n-k+1}\|_{2}  \\
    & \le \|E_0\|_{2} + \max_{k} \|A^{-1} R^{n-k+1}\|_{2}
  \end{aligned}
\end{equation}
}





\section*{Acknowledgments}
We would like to acknowledge the assistance of volunteers in putting
together this example manuscript and supplement.

\bibliographystyle{siamplain}
\bibliography{references}
\end{document}

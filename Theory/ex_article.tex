% SIAM Article Template
\documentclass[review,hidelinks,onefignum,onetabnum]{siamart220329}

% Information that is shared between the article and the supplement
% (title and author information, macros, packages, etc.) goes into
% ex_shared.tex. If there is no supplement, this file can be included
% directly.

\input{ex_shared}

% Optional PDF information
\ifpdf
\hypersetup{
  pdftitle={An Example Article},
  pdfauthor={D. Doe, P. T. Frank, and J. E. Smith}
}
\fi

% The next statement enables references to information in the
% supplement. See the xr-hyperref package for details.

\externaldocument[][nocite]{ex_supplement}

% FundRef data to be entered by SIAM
%<funding-group specific-use="FundRef">
%<award-group>
%<funding-source>
%<named-content content-type="funder-name"> 
%</named-content> 
%<named-content content-type="funder-identifier"> 
%</named-content>
%</funding-source>
%<award-id> </award-id>
%</award-group>
%</funding-group>

\begin{document}

\maketitle

% REQUIRED
\begin{abstract}
This is an example SIAM \LaTeX\ article. This can be used as a
template for new articles.  Abstracts must be able to stand alone
and so cannot contain citations to the paper's references,
equations, etc.  An abstract must consist of a single paragraph and
be concise. Because of online formatting, abstracts must appear as
plain as possible. Any equations should be inline.
\end{abstract}

% REQUIRED
\begin{keywords}
example, \LaTeX
\end{keywords}

% REQUIRED
\begin{MSCcodes}
68Q25, 68R10, 68U05
\end{MSCcodes}

\section{Introduction}
We study
\begin{equation}
  \frac{\partial u}{\partial t} +  (-\Delta)^{\frac{\alpha}{2}} u = f(x,t), \quad x \in \Omega, t\in (0,T].
\end{equation}
\begin{equation}
  D_t^\gamma u + (-\Delta)^{\frac{\alpha}{2}} u = f(x,t), \quad x \in \Omega, t\in (0,T].
\end{equation}
where
\begin{equation}
  \begin{aligned}
    D_t^\gamma u(x, t) = \frac{1}{\Gamma(1-\gamma)} \int_0^t \frac{\partial u(x, s)}{\partial s} (t-s)^{-\gamma} ds
  \end{aligned}
\end{equation}
\begin{equation}
  \begin{aligned}
    (-\Delta)^{\frac{\alpha}{2}} u(x,t) = \frac{1}{2\cos(\alpha\pi/2) \Gamma(2-\alpha)} \int_{0}^{2L} u(y, t) |x-y|^{1-\alpha} dy
  \end{aligned}
\end{equation}
where $\gamma \in (0,1)$, $\alpha \in (1,2)$.


% We will prove the convergence of this scheme.
\section{Regularity of the solution}
For the space-time fractional diffusion equation, it was first assumed that the solution regularity
satisfies
\begin{subequations} \label{eq:regularity}
  \begin{equation} \label{subeq:regularity-on-x}
    \left|\frac{\partial^{l} u}{\partial t^l}(x,t)\right| \le C(1+t^{\gamma-l}) \quad for \quad l = 0,1,2,
  \end{equation}
  \begin{equation} \label{subeq:regularity-of-Au-on-x}
    \left|\frac{\partial^{l} }{\partial x^l} (-\Delta)^{\alpha/2} u(x,t) \right| \le C \delta(x)^{-\alpha/2-l} \quad for \quad l = 0,1,2,
  \end{equation}
  \begin{equation} \label{subeq:regularity-on-t}
    \left|\frac{\partial^{l} u}{\partial x^l}(x,t)\right| \le C\delta(x)^{\alpha/2-l} \quad for \quad l = 0,1,2,3,4,
  \end{equation}
\end{subequations}
for all $(x,t) \in (0, 2L)\times (0,T]$.
\begin{remark}
  \eqref{subeq:regularity-of-Au-on-x} can be derived from \eqref{subeq:regularity-on-x} by
  \begin{equation*}
    \begin{aligned}
      I^{2-\alpha} u(x,t) &= \int_{0}^{x/2} + \int_{L+x/2}^{2L} u(y,t) \frac{|x-y|^{1-\alpha}}{\Gamma(2-\alpha)} dy \\
      &\quad + \int_{0}^{x/2} \left(u(x-z,t) + u(x+z,t) \right) \frac{z^{1-\alpha}}{\Gamma(2-\alpha)} dy \\
      &\quad + \int_{x+x/2}^{L+x/2} u(y,t) \frac{|y-x|^{1-\alpha}}{\Gamma(2-\alpha)} dy
    \end{aligned}
  \end{equation*}
\end{remark}


% \begin{lemma}
%   If
%   \begin{equation}
%     \left|\frac{\partial^{l} u}{\partial x^{l}}\right| \le \delta(x)^{\alpha/2-l}, \quad x \in \Omega,
%   \end{equation}
%   then
%   \begin{equation}
%     \left|-\frac{d^{l}}{dx^{l}} I^{2-\alpha} u \right| \le C \frac{\kappa_\alpha(\alpha-1)}{\Gamma(3-\alpha)} \delta(x)^{-\alpha/2 - (l-2)}, \quad l=1,2,3,4, \quad x \in \Omega,
%   \end{equation}
%   where $C$ is independent of $\alpha$.
% \end{lemma}
% \begin{proof}
%   \begin{equation}
%     \begin{aligned}
%       \frac{d^{l}}{dx^{l}} I^{2-\alpha} u &= \frac{\kappa_\alpha}{\Gamma(2-\alpha)} \Biggl(
%         (1-\alpha)\cdots(1-(l-1)-\alpha) \left(\int_0^{x/2} u(y) (x-y)^{1-l-\alpha} dy + (-1)^{l} \int_{T+x/2}^{2T} u(y) (y-x)^{1-l-\alpha} dy\right)  \\
%         &\quad + \sum_{k=1}^{l-1} (1-\alpha)\cdots(1-(k-1)-\alpha) \left(\frac{\partial ^{l-1-k}}{\partial x^{l-1-k}} u(x/2) \left(\frac{x}{2}\right)^{1-k-\alpha} - (-1)^{k}\frac{\partial ^{l-1-k}}{\partial x^{l-1-k}} u(T+x/2) \left(T+\frac{x}{2}\right)^{1-k-\alpha}   \right) \\
%         &\quad + (1-\alpha) \left(\int_{x/2}^{x} \frac{\partial ^{l-1}}{\partial x^{l-1}} u(y) (x-y)^{-\alpha} - \int_{x}^{T+x/2} \frac{\partial ^{l-1}}{\partial x^{l-1}} u(y) (y-x)^{-\alpha} \right)
%         \Biggl)
%     \end{aligned}
%   \end{equation}
%   Set $x\le T$, we have
%   \begin{equation*}
%     \left|\int_0^{x/2} u(y) (x-y)^{1-l-\alpha} dy\right| 
%     \le 2^{\alpha+l-1} \int_0^{x/2} y^{\alpha/2} x^{1-l-\alpha} dy
%     \simeq x^{-\alpha/2 - (l-2)}
%   \end{equation*}
%   \begin{equation*}
%     \frac{\partial ^{l-1-k}}{\partial x^{l-1-k}} u(x/2) \left(\frac{x}{2}\right)^{1-k-\alpha}
%     \lesssim  x^{-\alpha/2 - (l-2)}
%   \end{equation*}
%   \begin{equation*}
%     \begin{aligned}
%       &\int_{x/2}^{x} \frac{\partial ^{l-1}}{\partial x^{l-1}} u(y) (x-y)^{-\alpha} - \int_{x}^{T+x/2} \frac{\partial ^{l-1}}{\partial x^{l-1}} u(y) (y-x)^{-\alpha}  \\
%       & \quad = \int_{0}^{x/2} \left(\frac{\partial ^{l-1}}{\partial x^{l-1}} u(x-z)- \frac{\partial ^{l-1}}{\partial x^{l-1}} u(x+z)\right) z^{-\alpha} dz       - \int_{x+x/2}^{T+x/2} \frac{\partial ^{l-1}}{\partial x^{l-1}} u(y) (y-x)^{-\alpha} dy
%     \end{aligned}
%   \end{equation*}
%   While
%   \begin{equation*}
%     \begin{aligned}
%       &\left|\int_{0}^{x/2} \left(\frac{\partial ^{l-1}}{\partial x^{l-1}} u(x-z)- \frac{\partial ^{l-1}}{\partial x^{l-1}} u(x+z)\right) z^{-\alpha} dz \right|  \\
%       &\quad =  2\int_{0}^{x/2} \frac{\partial ^{l}}{\partial x^{l}} u(\xi)  z^{1-\alpha}  dz 
%       \le \left(\frac{x}{2}\right)^{\alpha/2-l} 2\int_{0}^{x/2} z^{1-\alpha}  dz \simeq \frac{1}{2-\alpha} x^{-\alpha/2 - (l-2)}
%     \end{aligned}
%   \end{equation*}
%   \begin{equation*}
%     \begin{aligned}
%       &\int_{x+x/2}^{T+x/2} \frac{\partial ^{l-1}}{\partial x^{l-1}} u(y) (y-x)^{-\alpha} dy \\
%       &\quad \le \int_{x+x/2}^{T+x/2} y^{\alpha/2-(l-1)} y^{-\alpha} dy
%       \lesssim  \begin{cases}
%         \frac{1}{2-\alpha} T^{-\alpha/2 - (l-2)}, & l=1, \\
%         \frac{1}{\alpha/2+l-1} x^{-\alpha/2 - (l-2)} , & l=2,3,4.
%       \end{cases}
%     \end{aligned}
%   \end{equation*}
%   % \begin{equation}
%   %   \begin{aligned}
%   %     \left|\frac{d^{l}}{dx^{l}} I^{2-\alpha} u(x) \right|
%   %     &\le \frac{\kappa_\alpha (\alpha-1)}{\Gamma(2-\alpha)} \Biggl(\cdots \Biggl)
%   %   \end{aligned}
%   % \end{equation}
% \end{proof}




\section{Numerical scheme}
\subsection{Discretisation of $(-\Delta)^{\frac{\alpha}{2}}$ on Graded Mesh}
\subsection{Discretisation of $D_t^{\gamma}$ on a General Mesh}

Consider the temporal mesh $0 = t0 < t1 < t2 < · · · < t_M = T$ . Set $\tau_j := t_j - t_{j-1}$ for $j = 1, . . . , M$.
\cite{Stynes:2019}

On this mesh, we discretise $D_t^\gamma v$ for $v \in C[0, T ] \cap C^3(0, T ]$.
\begin{equation}
  \begin{aligned}
    \delta_t^\gamma v(t_{k+\sigma}) &= \sum_{j=0}^{k} g_{k,j} \left( v(t_{j+1} - v(t_j)) \right) \\
    &= g_{k,k} v(t_{k+1}) - \sum_{j=1}^{k} \left(g_{k,j} - g_{k,j-1}\right)  v(t_j) - g_{k,0} v(t_0)
  \end{aligned}
\end{equation}
\begin{equation}
  \delta_t^\gamma U^{k+1} = g_{k,k} U^{k+1} - \sum_{j=1}^{k}  \left( g_{k,j} - g_{k,j-1}  \right)U^j - g_{k,0} U^0
\end{equation}

Implicity scheme: Let $\tau = \frac{T}{M}$, $U^{n}, F^{n} \in \mathbb{R}^{2N-1}$,
\begin{equation}
  \frac{U^{n+1} - U^{n} }{\tau} + AU^{n+1} = F^{n+1}.
\end{equation}
Then $E^{n} = U^{n} - \hat{U}^n \in \mathbb{R}^{2N-1}$,
\begin{equation}
  (I+\tau A)E^{n+1} = E^{n} + \tau R^{n+1}.
\end{equation}

\begin{equation}
  \delta_t^\gamma U^{n+1} + AU^{n+1} = F^{n+1}
\end{equation}
\begin{equation}
  \begin{aligned}
    E^{n} &= (I+\tau A)^{-1}E^{n-1} + (I+\tau A)^{-1}\tau R^{n} \\
    &= (I+\tau A)^{-n}E^{0} + \sum_{k=1}^{n} (I+\tau A)^{-k} \tau R^{n-k+1}
  \end{aligned}
\end{equation}

\section{Property of $A$ and $g_{k,j}$}
\begin{lemma}
  The stiffness matrix $A$ has the following properties:
  \begin{enumerate}
    \item The eigenvalues of $A$ are positive real numbers.
    \item $A$ is positive definite, which means that the eigenvalues of $\frac{A+A^T}{2}$ are positive.
    \item The eigenvectors of $A$ are orthogonal in space where $\langle u,v \rangle := uHv$, 
    where $H := \text{diag}\left( \frac{h_i+h_{i+1}}{2} \right)$.
    \item $(I+\tau A)^{-1} > O$ for any $\tau > 0$.
  \end{enumerate}
\end{lemma}
\begin{proof}
  Since
  \begin{equation}
    A = H^{-1} D = H^{-1/2} H^{-1/2} D H^{-1/2} H^{1/2} ,
  \end{equation}
  where $H^{-1/2} D H^{-1/2}$ is symmetric positive definite, $H^{-1/2} D H^{-1/2} = U\Lambda U^T$.
  Thus,
  \begin{equation}
    A = H^{-1/2} U \Lambda U^T H^{1/2} = (H^{-1/2} U) \Lambda (H^{-1/2} U)^{-1}.
  \end{equation}
  
  The eigenvectors of $A$ form an orthogonal basis of the Hilbert space defined by $\langle u,v \rangle := uHv$. 
  Let $v_i = H^{-1/2} u_i$ be an eigenvector of $A$ with eigenvalue $\lambda_i$.
\end{proof}
\textcolor{red}{We need to prove $\lambda_1 > c$ for some positive constant $c$.}



\section{truncation error}


\section{Convergence}
% We prove the shceme in convergence in the meaning of $L^2$ norm.
% $\|v\|_{2,h} := v_i^T H v_i$ is bounded.


Assume that $|R^{n}| \le C h^{\min\{r\alpha/2, 2\}}(x_i^{-\alpha} + (2T-x_i)^{-\alpha}) + C(r-1) h^2 (T - \delta(x_i) + h_N)^{1-\alpha} + C \tau^{?}$
\begin{equation}
  \begin{aligned}
    (I+\tau A)^{-k} \tau R^{n-k+1} &= (\tau A) (I+\tau A)^{-k} (\tau A)^{-1} \tau R^{n-k+1} \\
    &= (\tau A) (I+\tau A)^{-k} (A^{-1} R^{n-k+1})
  \end{aligned}
\end{equation}
Suppose that
\begin{equation}
  \begin{aligned}
    |R^{n}| &\le |R| \\
    &:= C h^{\min\{r\alpha/2, 2\}}(x_i^{-\alpha} + (2T-x_i)^{-\alpha}) \\
    & \quad+ C(r-1) h^2 (T - \delta(x_i) + h_N)^{1-\alpha} + C \tau^{?}
  \end{aligned}
\end{equation}
Since $0 < A^{-1} R \le C h^{\min}$,
% Depose it by the basis $v_i$, we have
\begin{equation}
  \begin{aligned}
    |(I+\tau A)^{-k} \tau R^{n-k+1}| 
    \le (I+\tau A)^{-k} \tau R 
    = \tau A (1+\tau A)^{-k} A^{-1} R
  \end{aligned}
\end{equation}
Then
\begin{equation}
  \begin{aligned}
    |E^{n}| &\le |(I+\tau A)^{-n}E^{0}| + \sum_{k=1}^{n} \tau A (1+\tau A)^{-k} A^{-1} R  \\
    &= |(I+\tau A)^{-n}E^{0}| + \left(I - (I+\tau A)^{-n}\right) A^{-1} R.
  \end{aligned}
\end{equation}
Since $A$ is diagonally dominant, $\|(I+\tau A)^{-1} E\|_\infty \le \|E\|_\infty$, we have
\begin{equation}
  \|E^{n}\|_\infty \le \|E_{0}\|_\infty + 2 \|A^{-1} R\|_\infty.
\end{equation}

\textcolor{red}{
\begin{lemma}
  $A^{-1} R$ is bounded by $C \left(h^{\min\{r\alpha/2, 2\}} + \tau^{?}\right)$, where $C$ is a constant independent of $h, \alpha$.
\end{lemma}
}


\section{Caputo-Riesz}
$\gamma\in (0,1)$

\begin{equation}
  \begin{aligned}
    D_N^\gamma u(x, t_n) 
    &= \sum_{k=1}^{n} \frac{1}{\Gamma(2-\gamma)}  \left( u(x,t_k) -  u(x,t_{k-1})\right) \frac{(t_n-t_{k-1})^{1-\gamma} - (t_n-t_k)^{1-\gamma}}{\tau_k} \\
    &= d_{n,n} u(x, t_n) - \sum_{k=1}^{n-1} (d_{n, k+1} - d_{n,k}) u(x, t_{k}) - d_{n,1} u(x, t_0),
  \end{aligned}
\end{equation}
where
\begin{equation}
  d_{n,k} = \frac{(t_n-t_{k-1})^{1-\gamma} - (t_n-t_{k})^{1-\gamma}}{\Gamma(2-\gamma)\tau_{k}} \quad \text{for} \quad 1\le k \le n \quad \text{and} \quad d_{n,0} = 0,
  % \simeq (1-\gamma)(t_n-t_{k-1})^{-\gamma},
\end{equation}
$\quad d_{n,n} = \frac{\tau_n^{-\gamma}}{\Gamma(2-\gamma)}, \quad d_{n,k+1} \ge d_{n,k}$.

Numerical scheme:
\begin{equation}
  D_N^\gamma U^n + A U^{n} = F^{n}
\end{equation}
We have
\begin{equation}
  \begin{aligned}
    \left(d_{n,n} I + A \right) E^n = \sum_{k=1}^{n-1} (d_{n, k+1} - d_{n,k}) E^{k} + d_{n,1} E^0 + R^{n}
  \end{aligned}
\end{equation}



Define the matrices $\Theta_{n,j}$, for $n = 1, 2 . . . , N$ and $j = 0, 1, 2, . . . , n - 1$ by
\begin{equation} \label{eq:def_Theta}
  \begin{gathered}
    \Theta_{n,n} = (d_{n,n} + A)^{-1} , \quad \Theta_{0,0} = I, \quad
    \Theta_{n,j} = \sum_{k=j}^{n-1} (d_{n, k+1} - d_{n,k})\Theta_{n,n} \Theta_{k,j}. %\quad 1\le j <n 
    % \Theta_{n,0} = d_{n,1} (d_{n,n} + A)^{-1} + \sum_{k=1}^{n-1} (d_{n, k+1} - d_{n,k})(d_{n,n} + A)^{-1} \Theta_{k,0}
  \end{gathered}
\end{equation}
Observe that $\Theta_{n,j} > O$ for all $n, j$.

\begin{lemma}
  \begin{equation}
    \begin{aligned}
      E^n = \sum_{j=1}^{n} \Theta_{n,j} R^{j} + \Theta_{n,0} E^{0}
    \end{aligned}
  \end{equation}
\end{lemma}

\begin{lemma}
  Let the parameter $\beta$ satisfy $\beta\le r_t \gamma$. Then for $n=1,2,..., N$, one has
  \begin{equation} \label{eq:sum_Theta}
    \sum_{j=1}^n \Theta_{n,j} < A^{-1}
    % \sum_{j=1}^n j^{-\beta} \Theta_{n,j} \le A^{-1}
  \end{equation}
\end{lemma}
\begin{proof}
  Use induction on $n$. When $n=1$, then
  $
    \sum_{j=1}^1  \Theta_{1,j} = \Theta_{1,1} < A^{-1}
  $.
  Next, assume that \eqref{eq:sum_Theta} holds for $k=1, 2,..., m-1(2\le m \le N)$. 
  We want to prove \eqref{eq:sum_Theta} for $n=m$.
  Invoking \eqref{eq:def_Theta} and interchanging the order of summation,
  \begin{equation*}
    \begin{aligned}
      \sum_{j=1}^m  \Theta_{m,j}
      &=  \Theta_{m,m} + \sum_{j=1}^{m-1}  \sum_{k=j}^{m-1} (d_{m, k+1} - d_{m,k})\Theta_{m,m} \Theta_{k,j} \\
      &= \Theta_{m,m} + \sum_{k=1}^{m-1} (d_{m, k+1} - d_{m,k})\Theta_{m,m}  \sum_{j=1}^{k} \Theta_{k,j} \\
      & \le  \Theta_{m,m} + \sum_{k=1}^{m-1} (d_{m, k+1} - d_{m,k})\Theta_{m,m}   A^{-1} \\
      &= \Theta_{m,m} + (d_{m, m} - d_{m,1})\Theta_{m,m}   A^{-1} \\
      &= A^{-1} - d_{m,1} \Theta_{m,m}   A^{-1} < A^{-1}
    \end{aligned}
  \end{equation*}
\end{proof}






\section*{Acknowledgments}
We would like to acknowledge the assistance of volunteers in putting
together this example manuscript and supplement.

\bibliographystyle{siamplain}
\bibliography{references}
\end{document}
